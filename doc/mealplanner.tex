\documentclass[a4paper]{article}

%% Language and font encodings
\usepackage[english]{babel}
\usepackage[utf8x]{inputenc}
\usepackage[T1]{fontenc}
\usepackage{amsthm}
\usepackage{textcomp}

%% Sets page size and margins
\usepackage[a4paper,top=3cm,bottom=2cm,left=2cm,right=2cm,marginparwidth=1.75cm]{geometry}

%% Useful packages
\usepackage{amsmath}
\usepackage{graphicx}
\usepackage[colorinlistoftodos]{todonotes}
\usepackage[colorlinks=true, allcolors=blue]{hyperref}
\usepackage{indentfirst}

% set up BNF generator
\usepackage{syntax}
\setlength{\grammarparsep}{10pt plus 1pt minus 1pt}
\setlength{\grammarindent}{10em}

\title{Meal Planning}
\author{Rob Moore - started 2017.01.26}

\begin{document}
\maketitle

\section{Solution-independent Specification}

\subsection{Daily specification}

\begin{itemize}
  \item 3 meals
  \item 3000 calories
  \item 360g carbohydrate
  \item 100g protein
  \item 130g fat
  \item 600g food total
  \item Eat less than £10 worth of food NB. This will change once I find what a reasonable value for it is.
\end{itemize}

\subsection{Meal specification}

Breakfast:
\begin{itemize}
  \item 70g carbohydrate
  \item 10g protein
  \item 10g fat
\end{itemize}

Lunch:
\begin{itemize}
  \item 80g carbohydrate
  \item 40g protein
  \item 50g fat
\end{itemize}

Dinner:
\begin{itemize}
  \item 50g carbohydrate
  \item 50g protein
  \item 70g fat
\end{itemize}

Additionally, all meals must have 3 different solid foods in them. Milk 
is considered a given for every meal, and hence the fat and protein 
values may be lower than they actually need to be. Furthermore I intend 
to be snacking throughout the day to keep calories high, so this is 
another source of leeway for the solutions.

Since at this stage the solutions are only suggestions, they need not be 
exhaustive, and I can decide for myself from the results what to eat.

\section{Required Function of the Program}

\subsection{Meal}

A user of the program must be able to request suggestions for a meal by giving it's name 
to the program on the command line, such as:

\begin{verbatim}
  ./mealplanner breakfast
\end{verbatim}

When a meal is specified, the following things should be shown:

\begin{itemize}
  \item List of foods for the meal
  \item Number of calories(N/A) in the meal
  \item Macronutrient(g) breakdown of the given meal
  \item Number of grams of food in the meal in total
  \item Cost(£) of buying each item in the meal from the shop
\end{itemize}

The foods returned must conform to all of the specification 
requirements. Furthermore, it must check the appropriateness of each 
food ie. Milk, cornflakes and toast for dinner is not a good suggestion.

\subsection{Day}

Furthermore, if the program is called without arguments, it should 
behave as though breakfast, lunch and dinner were all requested, giving 
meals for the whole day.

When giving results in this form, the meal plan must also be checked for 
conformity to daily requirements for total(g), macronutrient(g) and 
calorie(N/A) consumption.

\section{Algorithm}

A simple O($n^{3}$) search with optimisations at intermediate steps suffices 
for this problem.

I consider this to be okay because a local shop inventory database is 
never going to be large enough that this gets out of hand.

\section{Table}

\subsection{Spending}

Information about things I buy at the shops, rather than what I cook in the 
kitchen.

\begin{itemize}
  \item name (symbol)
  \item boughtfrom (symbol)
  \item price(£) (float)
  \item nServings (int)
  \item servingDescription (string)
  \item inmeals (bool list)
\end{itemize}

\subsection{GivenStats}

Statistics about the food I can see straight from the packet of food.

\begin{itemize}
  \item name (symbol)
  \item nGrams (int)
  \item calsP100g (int)
  \item gcarbsP100g (int)
  \item gproteinP100g (int)
  \item gfatP100g (int)
\end{itemize} 
  
\subsection{Nutrition}

Information to do with a food's nutritional value at serving time.

\begin{itemize}
  \item name (symbol)
  \item gtotalPserving (float)
  \item calsPserving (float)
  \item gcarbsPserving (float)
  \item gproteinPserving (float)
  \item gfatPserving (float)
\end{itemize}

\end{document}
