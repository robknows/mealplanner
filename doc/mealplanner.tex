\documentclass[a4paper]{article}

%% Language and font encodings
\usepackage[english]{babel}
\usepackage[utf8x]{inputenc}
\usepackage[T1]{fontenc}
\usepackage{amsthm}
\usepackage{textcomp}

%% Sets page size and margins
\usepackage[a4paper,top=3cm,bottom=2cm,left=2cm,right=2cm,marginparwidth=1.75cm]{geometry}

%% Useful packages
\usepackage{amsmath}
\usepackage{graphicx}
\usepackage[colorinlistoftodos]{todonotes}
\usepackage[colorlinks=true, allcolors=blue]{hyperref}
\usepackage{indentfirst}

% set up BNF generator
\usepackage{syntax}
\setlength{\grammarparsep}{10pt plus 1pt minus 1pt}
\setlength{\grammarindent}{10em}

\title{Meal Planning}
\author{Rob Moore - started 2017.01.26}

\begin{document}
\maketitle

\section{What the program does}

\subsection{Meal nutrient specification}

Breakfast:
\begin{itemize}
  \item 70g carbohydrate
  \item 10g protein
  \item 10g fat
\end{itemize}

Lunch:
\begin{itemize}
  \item 75g carbohydrate
  \item 40g protein
  \item 50g fat
\end{itemize}

Dinner:
\begin{itemize}
  \item 50g carbohydrate
  \item 50g protein
  \item 50g fat
\end{itemize}

Additionally, all meals must have the right combination of foods in them, as 
specified below:


Breakfast:
\begin{itemize}
  \item 1 cereal 
  \item 1 side or 1 bread.
\end{itemize}

Lunch:
\begin{itemize}
  \item 1 meat
  \item 1 healthy
  \item One of 1 staple, 1 bread or 1 canned 
  \item Optionally one of 1 dessert, 1 side or 1 dairy.
\end{itemize}

Dinner:
\begin{itemize}
  \item 1 meat
  \item 1 healthy
  \item One of 1 staple, 1 bread or 1 canned 
  \item One of 1 dessert, 1 side or 1 dairy.
\end{itemize}

\section{Use of the Program}

\subsection{Meal}

A user of the program must be able to request suggestions for a meal by giving it's name 
to the program on the command line, such as:

\begin{verbatim}
  ./mealplanner breakfast
\end{verbatim}

When a meal is specified, the following things should be shown:

\begin{itemize}
  \item List of foods for the meal
  \item The shops I need to visit to buy the food
  \item Number of calories in the meal
  \item Macronutrient(g) breakdown of the given meal
  \item Number of grams of food in the meal in total
  \item Cost(£) of buying each item in the meal from the shop
\end{itemize}

The foods returned must conform to all of the specification 
requirements. Furthermore, it must check the appropriateness of each 
food ie. Cornflakes and toast for dinner is not a good suggestion.

\section{Algorithm}

A simple O($n^{3}$) search with optimisations at intermediate steps suffices 
for this problem.

I consider this to be okay because a local shop inventory database is 
never going to be large enough that this complexity gets out of hand.

\section{Table}

\subsection{Spending}

Information about things I buy at the shops, rather than what I cook in the 
kitchen.

\begin{itemize}
  \item name (symbol)
  \item boughtfrom (symbol)
  \item price(£) (float)
  \item nServings (int)
  \item servingDescription (string)
  \item inmeals (bool list)
  \item foodtype (symbol)
\end{itemize}

\subsection{GivenStats}

Statistics about the food I can see straight from the packet of food.

\begin{itemize}
  \item name (symbol)
  \item nGrams (int)
  \item calsP100g (int)
  \item gcarbsP100g (int)
  \item gproteinP100g (int)
  \item gfatP100g (int)
\end{itemize} 
  
\subsection{Nutrition}

Information to do with a food's nutritional value at serving time.

\begin{itemize}
  \item name (symbol)
  \item gtotalPserving (float)
  \item calsPserving (float)
  \item gcarbsPserving (float)
  \item gproteinPserving (float)
  \item gfatPserving (float)
\end{itemize}

\subsection{Cost}

Normalised price statistics about a food.

\begin{itemize}
  \item name (symbol)
  \item pricePserving (float)
  \item gramsPpound (float)
  \item caloriesPpound (float)
\end{itemize}

\end{document}
